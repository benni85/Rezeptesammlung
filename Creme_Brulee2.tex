\documentclass[a6paper, twoside]{report}
\usepackage[a6paper, top=7mm, inner=10mm, right=10mm, bottom=10mm, landscape]{geometry}
\usepackage[ngerman]{babel}
\usepackage[utf8]{inputenc}
\usepackage{multicol}
\pagestyle{plain}



\parindent0pt	

\pagestyle{empty}
\begin{document}

\textbf{{\LARGE Creme Brulee}}%%%%%%%%%%%%%%%%%%%%%%%%%%%%%%%%%%%%%%%%%%%%%%%%%%%%%%%%%%%%%%%%%%%%%%%%


\hrulefill
\vspace*{\fill}
\begin{multicols}{2}	


\begin{itemize}
\item 1kg    Creme Fraiche
\item 20 g     Vanillezucker
\item 120 g    Weisser Zucker
\item 1     Schale einer Orange
\item 2    Eigelb
\item 2     Eier
\end{itemize}

\end{multicols}

\vspace{1cm}
\begin{center}
[Zutaten für 10 Portionen]
\end{center}

\vfill
\newpage
\textbf{{\LARGE Zubereitung}}%%%%%%%%%%%%%%%%%%%%%%%%%%%%%%%%%%%%%%%%%%%%%%%%%%%%%%%%%%%%%%%%%%%%%%%%

\hrulefill

\vspace*{\fill}
\begin{multicols}{2}
\columnseprule1pt
%Celsiuszeichen: 	$^\circ$


Sämtliche Zutaten in ein hohes Gefäß geben und mit dem Stabmixer gut aufschlagen.\\
Die Mischung auf 10 Förmchen verteilen.\\
 
Ein tiefes Backblech oder eine entsprechende Auflaufform (die Förmchen für die Creme müssen alle hineinpassen)
1-2 cm tief mit warmem Wasser gefüllt im Backofen \textbf{auf 180$^\circ$C} vorheizen.\\

In das Wasserbad im Backofen ein Küchentuch legen und die Förmchen darauf stellen. 
Gegebenenfalls noch etwas kochendes Wasser in das Wasserbad nachfüllen, das Wasser sollte etwa so hoch stehen, 
wie die Masse in den Förmchen. 
Den Ofen schließen und \textbf{nach 25 Minuten} mit dem Zeigefinger eine leichte Druckprobe am vorderstem Förmchen machen. 
Wenn die Creme sich nun "puddingartig" anfühlt, dann ist sie fertig.\\
Kalt stellen und die Förmchen mit Klarsichtfolie abdecken bis sie gebraucht werden.\\ 
Erst vor dem Servieren mit braunem Zucker bestreuen und den Zucker mit einem Bunsenbrenner karamellisieren.\\


Variationen:\\ 
Statt der geriebenen Orangenschale geht auch geriebene Limettenschale.
Auch ein wenig Espresso in der Creme ist zu empfehlen.


\end{multicols}
\vfill
\end{document}