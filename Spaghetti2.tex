\documentclass[a6paper, twoside]{report}
\usepackage[a6paper, top=7mm, inner=10mm, right=10mm, bottom=10mm, landscape]{geometry}
\usepackage[ngerman]{babel}
\usepackage[utf8]{inputenc}
\usepackage{multicol}
\pagestyle{plain}



\parindent0pt	

\pagestyle{empty}
\begin{document}

\textbf{{\LARGE Spaghetti con Piennolo}}%%%%%%%%%%%%%%%%%%%%%%%%%%%%%%%%%%%%%%%%%%%%%%%%%%%%%%%%%%%%%%%%%%%%%%%%


\hrulefill
\vspace*{\fill}
\begin{multicols}{2}	


\begin{itemize}
\item 600 g Kirschtomaten (ideal: Piennolo-Tomaten)
\item 2 Knoblauchzehen
\item Olivenöl
\item Salz
\item 1 Bund Basilikum
\item Parmesan
\item 400 g Spaghetti
\end{itemize}

\end{multicols}
\vfill
\newpage
\textbf{{\LARGE Zubereitung}}%%%%%%%%%%%%%%%%%%%%%%%%%%%%%%%%%%%%%%%%%%%%%%%%%%%%%%%%%%%%%%%%%%%%%%%%

\hrulefill

\vspace*{\fill}
\begin{multicols}{2}
\columnseprule1pt


Die geschnittenen Knoblauchzehen in Olivenöl für wenige Minuten anbraten, 
dann etwas Basilikum und die halbierten ungeschälten Tomaten zugeben.\\
Lange genug in der Pfanne kochen lassen bis die Tomaten ihren Saft abgegeben haben.\\
Die Spaghetti bis zur gewünschten Bissfestigkeit kochen und die Sauce mit den Tomaten drübergeben.\\ 

Mit dem Rest des Basilikums und dem geriebenen Parmesan servieren. 



\end{multicols}
\vfill
\end{document}