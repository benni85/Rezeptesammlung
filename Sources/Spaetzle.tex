\documentclass[a6paper, twoside]{report}
\usepackage[a6paper, top=7mm, inner=10mm, right=10mm, bottom=10mm, landscape]{geometry}
\usepackage[ngerman]{babel}
\usepackage[utf8]{inputenc}
\usepackage{multicol}
\pagestyle{plain}




\parindent0pt	

\pagestyle{empty}
\begin{document}

\textbf{{\LARGE Risotto alla Milanese}}%%%%%%%%%%%%%%%%%%%%%%%%%%%%%%%%%%%%%%%%%%TITEL
%%%%%%%%%%%%%%%%%%%%%%%%%%%%%%%%%%%%%%%%%%%%%%%%%%%%%%%%%%%%%UNTERTITEL
%			%%QUELLE als Kommentar

\hrulefill
\vspace*{\fill}
\begin{multicols}{2}	


\begin{itemize}
\item 150 g 	Reis 
\item 1		Schalotte
\item 750 ml 	Hühnerbrühe
\item 75 ml 	Weißwein
\item 40 g	Butter
\item $\frac{1}{2}$ EL Safranfäden
\item 50 g 	kalte Butter, gewürfelt
\item 40 g	Parmesan, gerieben
\end{itemize}
\end{multicols}
\vfill									%%Neue//

\vspace{2cm}
%
\begin{center}
%
[Zutaten für 2 Portionen]%%%%%%%%%%%%%%%%%%%%%%%%%%%%%%%%%%%%%%%%%%%%%%%MENGE
\end{center}


\vfill
\newpage
\textbf{{\LARGE Zubereitung}}%%%%%%%%%%%%%%%%%%%%%%%%%%%%%%%%%%%%%%%%%%%%%%%%

\hrulefill

\vspace*{\fill}
\begin{multicols}{2}
\columnseprule1pt

Die Bühe heisshalten ohne sie zu kochen.

%solfriggere
Die Schalotte kleinschneiden und sanft ohne zu bräunen 
in der Butter anschwitzen.\newline

%tostare
Den Reis zugeben und solange wenden bis jedes Korn mit Butter
benetzt ist.
Jetzt die Temperatur vorsichtig leicht erhöhen und den Wein
zugießen. Dieser muss ganz verdampfen bevor der Safran zugegeben wird.\newline

%cuocere
Kellenweise die Brühe zugeben und ca. 18 Minuten lang zu einem
cremigen, bissfesten Risotto kochen.
Permanent umrühren und  die Körner vom Rand und Boden schaben.
Währenddessen die Temperatur so regeln das die Brühe gerade eben kocht.
Ist die Brühe fast eingekocht die nächste Kelle zugeben.\newline
Ab der 14. Minute weniger Brühe zugeben damit der Reis nicht zu
flüssig wird.\newline
Am Ende die Temperatur drastisch verringern um die Masse für eine
Minute ruhen zu lassen.\newline

%Montecatura
Die Butterwürfel und den Parmesan unterrühren.



\end{multicols}
\vfill
\end{document}