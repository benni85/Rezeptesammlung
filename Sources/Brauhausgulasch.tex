\documentclass[a6paper, twoside]{report}
\usepackage[a6paper, top=7mm, inner=10mm, right=10mm, bottom=10mm, landscape]{geometry}
\usepackage[ngerman]{babel}
\usepackage[utf8]{inputenc}
\usepackage{multicol}
\pagestyle{plain}




\parindent0pt	

\pagestyle{empty}
\begin{document}

\textbf{{\LARGE Brauhaus-Gulasch}}%%%%%%%%%%%%%%%%%%%%%%%%%%%%%%%%%%%%%%%%%%TITEL
%%%%%%%%%%%%%%%%%%%%%%%%%%%%%%%%%%%%%%%%%%%%%%%%%%%%%%%%%%%%%UNTERTITEL
%		%%QUELLE als Kommentar

\hrulefill
\vspace*{\fill}
\begin{multicols}{2}	


\begin{itemize}
\item 500 g Gulasch vom Schwein
\item 2 EL Butterschmalz
\item $\frac{1}{2}$ TL Senf
\item 1 EL Tomatenmark
\item 1 große Zwiebel, gehackt
\item 1 Knoblauchzehe, gepresst
\item 1 Karotte, gerieben
\item 200 ml Franziskaner Weissbier \newline
naturtrüb
\item 600 ml Gemüsebrühe
\item 1 TL Salz
\item 1 TL Pfeffer, gemahlen
\item 1 TL Paprikapulver, edelsüß
\end{itemize}
\end{multicols}
\vfill									%%Neue//

\vspace{0.1 cm}
%
\begin{center}
%
[Zutaten für 4 Portionen]%%%%%%%%%%%%%%%%%%%%%%%%%%%%%%%%%%%%%%%%%%%%%%%MENGE
\end{center}


\vfill
\newpage
\textbf{{\LARGE Zubereitung}}%%%%%%%%%%%%%%%%%%%%%%%%%%%%%%%%%%%%%%%%%%%%%%%%

\hrulefill

\vspace*{\fill}
\begin{multicols}{2}
\columnseprule1pt

Das Butterschmalz in einer tiefen Pfanne erhitzen. \newline
Das Fleisch im Butterschmalz scharf anbraten und mit Salz, Pfeffer und Paprikapulver würzen.\newline

Wenn das Fleisch eine schöne Farbe angenommen hat und das Wasser verdampft ist, die gehackte Zwiebel, die gepresste
Knoblauchzehe und die geriebene Karotte, sowie den Senf und das Tomatenmark zugeben und kurz weiterbraten lassen.\newline

Anschließend mit dem Bier ablöschen und einkochen lassen. Jetzt mit der Gemüsebrühe auffüllen und mit Deckel \textbf{ca. 1 Stunde} köcheln lassen.\newline

Etwa 5 Min. vor Ende der Garzeit nach Bedarf mit etwas Soßenbinder andicken.

\end{multicols}
\vfill
\end{document}