\documentclass[a6paper, twoside]{report}
\usepackage[a6paper, top=7mm, inner=10mm, right=10mm, bottom=10mm, landscape]{geometry}
\usepackage[ngerman]{babel}
\usepackage[utf8]{inputenc}
\usepackage{multicol}
\pagestyle{plain}



\parindent0pt	

\pagestyle{empty}
\begin{document}

\textbf{{\LARGE Lokum}}%%%%%%%%%%%%%%%%%%%%%%%%%%%%%%%%%%%%%%%%%%%%%%%%%%%%%%%%%%%%%%%%%%%%%%%%


\hrulefill

Türkischer Honig
\vspace*{\fill}
\begin{multicols}{2}	


\begin{itemize}
\item 300 g festen Honig
\item 300 g Kristallzucker
\item 250 ml Wasser
\item 20 g Kakaofett (oder ersatzweise Kokosfett)
\item 3 Eiweiß 
\item 10 g Zucker (ca. 1 EL)
\item 150 g Geröstete halbierte Mandeln
\item Geröstete Haselnüsse
\item Kandierte, gewürfelte Früchte
\item Pistazien

\end{itemize}

\end{multicols}
\vfill
\newpage
\textbf{{\LARGE Zubereitung}}%%%%%%%%%%%%%%%%%%%%%%%%%%%%%%%%%%%%%%%%%%%%%%%%%%%%%%%%%%%%%%%%%%%%%%%%

\hrulefill

\vspace*{\fill}
\begin{multicols}{2}
\columnseprule1pt


\textbf{Backofen auf 200 Grad, Umluft 180 Grad}\\
Die Mandeln auf einem Backblech im Ofen etwa 8 Minuten rösten. 

Ein weiteres Backblech mit Backpapier auslegen. 
Den Honig in einem Topf zum Kochen bringen und das Eiweiß in die Metallschüssel 
geben und den Zucker mit einer Prise Salz einrühren. 
Die Masse mit dem Schneebesen leicht zu einem weichen Eischnee aufschlagen. In einem zweiten Topf den Zucker 
mit dem Wasser unter ständigem Rühren langsam zum Kochen bringen, 
dabei die Bildung von Zuckerkristallen vermeiden und bei etwa 140 Grad auf etwa die Hälfte zu einem Sirup reduzieren.\\
In einer Schüssel den Honig und die Zuckermischung unter ständigem Rühren vermengen.
Den Eischnee auf das Wasserbad setzten und die Zucker-Honigmischung langsam in einem dünnen Strahl unter ständigem Rühren unter den Eischnee schlagen.\\
Die Eiweißmasse schlagen, bis diese fest geworden ist und sich vom Schüsselrand löst.
Zum Schluss Kakaobutter, Früchte und Nüsse unterrühren. \\
Ein tiefes Blech mit gefettetem Backpapier auslegen.
Über Nacht auskühlen lassen. Lokum sollte nach dem Auskühlen mit einem Messer in Rauten geschnitten werden. 
Die Würfel können später in einer Schachtel mit Mehl oder Kokosnussraspeln aufbewahrt werden, 
damit sie nicht aneinander kleben.

\end{multicols}
\vfill
\end{document}