\documentclass[a6paper, twoside]{report}
\usepackage[a6paper, top=7mm, inner=10mm, right=10mm, bottom=10mm, landscape]{geometry}
\usepackage[ngerman]{babel}
\usepackage[utf8]{inputenc}
\usepackage{multicol}
\pagestyle{plain}




\parindent0pt	

\pagestyle{empty}
\begin{document}

\textbf{{\LARGE Pancakes}}%%%%%%%%%%%%%%%%%%%%%%%%%%%%%%%%%%%%%%%%%%TITEL
%%%%%%%%%%%%%%%%%%%%%%%%%%%%%%%%%%%%%%%%%%%%%%%%%%%%%%%%%%%%%UNTERTITEL
%		%%QUELLE als Kommentar

\hrulefill
\vspace*{\fill}
\begin{multicols}{2}	


\begin{itemize}
\item $\frac{3}{4}$ Tasse Mehl
\item $\frac{3}{4}$ Tasse Milch
\item 1 Ei
\item 1 gehäufter TL Backpulver
\item 1 Prise Salz
\item Butter
\end{itemize}
\end{multicols}
\vfill									%%Neue//

\vspace{2cm}
%
\begin{center}
%
[Zutaten für 2 Portionen]%%%%%%%%%%%%%%%%%%%%%%%%%%%%%%%%%%%%%%%%%%%%%%%MENGE
\end{center}


\vfill
\newpage
\textbf{{\LARGE Zubereitung}}%%%%%%%%%%%%%%%%%%%%%%%%%%%%%%%%%%%%%%%%%%%%%%%%

\hrulefill

\vspace*{\fill}
\begin{multicols}{2}
\columnseprule1pt

Alle Zutaten, bis auf die Butter, in einer Schüssel vermischen bis ein leicht flüssiger Teig entsteht.\newline

Etwas Butter in einer beschichteten Pfanne erhitzen. Dann mit einem Papiertuch das überschüssige Fett
wegwischen sodass die Pfanne leicht benetzt ist.\newline

Den Teig portionsweise mit einem Schöpfer in die Pfanne geben und auf beiden Seiten backen. Jeweils wenden
wenn sich auf der Oberseite die Bläschen öffnen.\newline

Je nach Geschmack noch etwas Butter auf die warmen Pancakes geben.

\end{multicols}
\vfill
\end{document}