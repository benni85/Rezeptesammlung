\documentclass[a6paper, twoside]{report}
\usepackage[a6paper, top=7mm, inner=10mm, right=10mm, bottom=10mm, landscape]{geometry}
\usepackage[ngerman]{babel}
\usepackage[utf8]{inputenc}
\usepackage{multicol}
\pagestyle{plain}



\parindent0pt	

\pagestyle{empty}
\begin{document}

\textbf{{\LARGE Sachertorte}}%%%%%%%%%%%%%%%%%%%%%%%%%%%%%%%%%%%%%%%%%%%%%%%%%%%%%%%%%%%%%%%%%%%%%%%%
%die andere Variante ist besser

\hrulefill
\vspace*{\fill}
\begin{multicols}{2}	


Für den Teig:
\begin{itemize}
\item 120 g 	Blockschokolade (mind. 40\% Kakao)
\item 120 g 	Margarine
\item 6  	Eigelb
\item 100 g 	Puderzucker
\item 6  	Eiweiß
\item 80 g 	Zucker
\item 80 g 	Mehl
\item40 g 	Maisstärke
\end{itemize}

Für die Füllung:
\begin{itemize}
\item Aprikosenmarmelade
\item Rum
\end{itemize}
Für die Glasur:
\begin{itemize}
\item Margarine
\item Schokolade 
\end{itemize}

\end{multicols}
\vfill
\newpage
\textbf{{\LARGE Zubereitung}}%%%%%%%%%%%%%%%%%%%%%%%%%%%%%%%%%%%%%%%%%%%%%%%%%%%%%%%%%%%%%%%%%%%%%%%%

\hrulefill

\vspace*{\fill}
\begin{multicols}{2}
\columnseprule1pt


Diese Menge reicht für eine Tortenform von 24 cm.
Schokolade und Margarine langsam schmelzen lassen, glatt rühren - überkühlen.
Eidotter und Puderzucker schaumig rühren und das Schokolade-Margarine-Gemisch langsam dazu. 
In einer anderen Schüssel Eiklar und Kristallzucker schaumig schlagen. 
Mehl und Maisstärke vermischen und abwechselnd mit dem Eischnee unter die Dotter-Schoko-Masse heben.
In einer gefetteten, bemehlten Form im vorgeheizten Ofen \textbf{ bei $180^\circ$C 10 Minuten} bei spaltbreit offener 
(Kochlöffel einklemmen) Türe backen. \textbf{Für 50 Minuten bei $140^\circ$C fertigbacken}.\\

Für die Füllung die Marmelade leicht anwärmen, mit einer Gabel schaumig rühren und ein bisschen Rum oder Cognac dazugeben.
Die Torte ein bis zweimal horizontal durchschneiden und mit Marmelade füllen.\\ 

Für die Glasur zu gleichen Teilen Schokolade (mind. 40\%) und Margarine schmelzen, glatt rühren und über die Torte gießen. 
Bei Zimmertemperatur erkalten lassen.



\end{multicols}
\vfill
\end{document}