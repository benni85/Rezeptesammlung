\documentclass[a6paper, twoside]{report}
\usepackage[a6paper, top=7mm, inner=10mm, right=10mm, bottom=10mm, landscape]{geometry}
\usepackage[ngerman]{babel}
\usepackage[utf8]{inputenc}
\usepackage{multicol}
\pagestyle{plain}



\parindent0pt	

\pagestyle{empty}
\begin{document}

\textbf{{\LARGE Tiramisu}}%%%%%%%%%%%%%%%%%%%%%%%%%%%%%%%%%%%%%%%%%%%%%%%%%%%%%%%%%%%%%%%%%%%%%%%%


\hrulefill
\vspace*{\fill}
\begin{multicols}{2}	


\begin{itemize}
\item 3 	Eier
\item 250 g 	Mascarpone
\item 120 g 	Löffelbiskuits
\item 1 Tasse starker Kaffee
\item 3 EL Zucker
\item 2 EL Amaretto oder Marsala
\item Kakaopulver 
\end{itemize}

\end{multicols}
\vfill
\newpage
\textbf{{\LARGE Zubereitung}}%%%%%%%%%%%%%%%%%%%%%%%%%%%%%%%%%%%%%%%%%%%%%%%%%%%%%%%%%%%%%%%%%%%%%%%%

\hrulefill

\vspace*{\fill}
\begin{multicols}{2}
\columnseprule1pt


Die Eier trennen, das Eigelb mit dem Zucker cremig schlagen und anschließend den Mascarpone unterrühren. \\
Das Eiweiß steif schlagen und vorsichtig unterheben.\\
 
Die Löffelbiskuits kurz in eine Mischung aus dem Alkohol und dem Kaffee eintunken oder beidseitig einpinseln.\\

Den Boden einer Glasschale mit einer Schicht Biskuits auslegen und diese dünn mit der Mascarponecreme bedecken.\\
Diese Schichtung fortsetzen bis alles aufgebraucht ist, am Ende mit einer Cremeschicht abschliessen.\\

Das Tiramisu dann für mindestens 3 Stunden kaltstellen (besser 12 Stunden). \\

Vor dem Verzehr dick mit Kakao bestäuben.


\end{multicols}
\vfill
\end{document}