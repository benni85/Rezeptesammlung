\documentclass[a6paper, twoside]{report}
\usepackage[a6paper, top=7mm, inner=10mm, right=10mm, bottom=10mm, landscape]{geometry}
\usepackage[ngerman]{babel}
\usepackage[utf8]{inputenc}
\usepackage{multicol}
\pagestyle{plain}




\parindent0pt	

\pagestyle{empty}
\begin{document}

\textbf{{\LARGE Schokomandeln}}%%%%%%%%%%%%%%%%%%%%%%%%%%%%%%%%%%%%%%%%%%TITEL
%%%%%%%%%%%%%%%%%%%%%%%%%%%%%%%%%%%%%%%%%%%%%%%%%%%%%%%%%%%%%UNTERTITEL
%		%%QUELLE als Kommentar

\hrulefill
\vspace*{\fill}
\begin{multicols}{2}	


\begin{itemize}
\item 100 g 	dunkle Kuvertüre
\item 30 g	Kakao
\item 200 g 	ungeschälte Mandeln	
\item 1 EL 	Staubzucker
\item 3 EL 	Whisky
\end{itemize}
\end{multicols}
\vfill									%%Neue//

\vspace{2cm}
%



\vfill
\newpage
\textbf{{\LARGE Zubereitung}}%%%%%%%%%%%%%%%%%%%%%%%%%%%%%%%%%%%%%%%%%%%%%%%%

\hrulefill

\vspace*{\fill}
\begin{multicols}{2}
\columnseprule1pt


Den Whisky mit Zucker und Mandeln mischen. Die Mandeln dann
auf einem mit Backpapier belegtem Blech flach verteilen und 
\textbf{ca. 10 Minuten bei 180$^\circ$ C} rösten.\newline

Anschließend die Mandeln gut abkühlen lassen.\newline
Die Kuvertüre schmelzen und die Mandeln portionsweise darin
wenden und flach und einzeln auf das Backblech legen.
Das Blech kühl stellen bis die Schokolade vollständig abgekühlt ist.\newline

Wenn die Glasur fest ist können die Mandeln im Kakao 
gewälzt werden.


\end{multicols}
\vfill
\end{document}