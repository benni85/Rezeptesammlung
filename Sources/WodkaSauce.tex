\documentclass[a6paper, twoside]{report}
\usepackage[a6paper, top=7mm, inner=10mm, right=10mm, bottom=10mm, landscape]{geometry}
\usepackage[ngerman]{babel}
\usepackage[utf8]{inputenc}
\usepackage{multicol}
\pagestyle{plain}




\parindent0pt	

\pagestyle{empty}
\begin{document}

\textbf{{\LARGE Wodka-Sauce}}%%%%%%%%%%%%%%%%%%%%%%%%%%%%%%%%%%%%%%%%%%TITEL
%%%%%%%%%%%%%%%%%%%%%%%%%%%%%%%%%%%%%%%%%%%%%%%%%%%%%%%%%%%%%UNTERTITEL
%Von Steffi			%%QUELLE als Kommentar

\hrulefill
\vspace*{\fill}
\begin{multicols}{2}	


\begin{itemize}
\item 3 Tassen Marinara-Sauce
\item 1 Tasse Wodka
\item $\frac{1}{2}$ Tasse Sahne
\item $\frac{1}{2}$ Tasse geriebener Parmesan
\item $\frac{1}{2}$ TL Salz
\item $\frac{1}{2}$ TL schwarzer Pfeffer
\end{itemize}
\end{multicols}
\vfill									%%Neue//



\vfill
\newpage
\textbf{{\LARGE Zubereitung}}%%%%%%%%%%%%%%%%%%%%%%%%%%%%%%%%%%%%%%%%%%%%%%%%

\hrulefill

\vspace*{\fill}
\begin{multicols}{2}
\columnseprule1pt

Sauce und Wodka in einer großen Pfanne über niedriger Hitze 
köcheln lassen, dabei permanent rühren.\newline

Die Sahne (auf Raumtemperatur temperiert) einrühren und weiterköcheln lassen bis alles gleichmäßig
erhitzt ist. Dabei ständig weiterrühren.\newline

Die Pfanne vom Herd nehmen und den Parmesan, das Salz und den
Pfeffer unterrühren.\newline

Nach Geschmack mit Salz und Pfeffer würzen.  

\end{multicols}
\vfill
\end{document}