\documentclass[a6paper, twoside]{report}
\usepackage[a6paper, top=7mm, inner=10mm, right=10mm, bottom=10mm, landscape]{geometry}
\usepackage[ngerman]{babel}
\usepackage[utf8]{inputenc}
\usepackage{multicol}
\pagestyle{plain}




\parindent0pt	

\pagestyle{empty}
\begin{document}

\textbf{\LARGE{Crème de caramel au beurre salé}}%%%%%%%%%%%%%%%%%%%%%%%%%%%%%%%%%%%%%%%%%%TITEL
%%%%%%%%%%%%%%%%%%%%%%%%%%%%%%%%%%%%%%%%%%%%%%%%%%%%%%%%%%%%%UNTERTITEL
%		%%QUELLE als Kommentar

\hrulefill



\vspace{0,5 cm}
\vspace*{\fill}
\begin{multicols}{2}	


\begin{itemize}
\item 200 g Zucker
\item 100 g Butter
\item 175 g Schlagsahne
\item 2 Prisen Fleur de sel
\end{itemize}
\end{multicols}
\vfill									%%Neue//

\vspace{1cm}
%
\begin{center}
%
[Zutaten für 2 Gläser]%%%%%%%%%%%%%%%%%%%%%%%%%%%%%%%%%%%%%%%%%%%%%%%MENGE
\end{center}



\vfill
\newpage
\textbf{{\LARGE Zubereitung}}%%%%%%%%%%%%%%%%%%%%%%%%%%%%%%%%%%%%%%%%%%%%%%%%

\hrulefill

\vspace*{\fill}
\begin{multicols}{2}
\columnseprule1pt


Den Zucker mit Zwei Esslöffeln Wasser karamellisieren bis er eine goldene Farbe angenommen hat.
Währenddessen die Masse nicht berühren oder umrühren.\newline

\textbf{Das Karamell darf nicht zu dunkel werden!}\newline

Die Butter wird in kleinen Stücken untergerührt, anschließend das Salz zugeben.\newline

Die Sahne erwärmen, aber nicht kochen, und dann in die Karamellmasse rühren.\newline
Die Creme dann noch 2 Minuten kochen.\newline
Im Kühlschrank hält sich die Creme für ca. einen Monat.









\end{multicols}
\vfill
\end{document}