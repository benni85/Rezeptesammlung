\documentclass[a6paper, twoside]{report}
\usepackage[a6paper, top=7mm, inner=10mm, right=10mm, bottom=10mm, landscape]{geometry}
\usepackage[ngerman]{babel}
\usepackage[utf8]{inputenc}
\usepackage{multicol}
\pagestyle{plain}




\parindent0pt	

\pagestyle{empty}
\begin{document}

\textbf{{\LARGE Haselnussplätzchen}}%%%%%%%%%%%%%%%%%%%%%%%%%%%%%%%%%%%%%%%%%%TITEL
%%%%%%%%%%%%%%%%%%%%%%%%%%%%%%%%%%%%%%%%%%%%%%%%%%%%%%%%%%%%%UNTERTITEL
%Geburtstagsgeschenk von Susanne		%%QUELLE als Kommentar

\hrulefill

Ein Geburtstagsgeschenk von Susanne Grimm
\vspace*{\fill}
\begin{multicols}{2}	

Für den Teig:
\begin{itemize}
\item 75 g 	Dinkelvollkornmehl
\item 125 g 	Weizenmehl
\item 40 g 	Rohrzucker
\item 60 g 	Zucker
\item 1 	Ei
\item 200 g	kalte Butter
\item 200 g 	Haselnusskerne
\item 50 g 	Mandeln
\item 1 Prise	Salz
\item 1 Prise	Zimt
\item 1 Prise	Bourbon-Vanille
\end{itemize}

Für die Verzierung: 
\begin{itemize}
\item 100 g	Nuss-Nougat
\item 10 g	Kokosöl 	
\end{itemize}
\end{multicols}
\vfill									%%Neue//




\vfill
\newpage
\textbf{{\LARGE Zubereitung}}%%%%%%%%%%%%%%%%%%%%%%%%%%%%%%%%%%%%%%%%%%%%%%%%

\hrulefill

\vspace*{\fill}
\begin{multicols}{2}
\columnseprule1pt



Butter schaumig schlagen, dann das Ei zugeben.

Nach und nach alle anderen Zutaten in die Masse rühren.\newline

Den Teig dann auf einer gemehlten Arbeitsfläche
Den Teig mit etwas Mehl zu einer Kugel formen und für 30 Minuten kaltstellen.\newline

Anschliessend den Teig in 6 Teile schneiden und aus jedem
eine ca 25 cm lange Rolle formen.
Die Rollen auf einem mit Backpapier belegtem Blech in 8 cm Abstand
aufreihen.\newline

Im Ofen bei \textbf{200$^\circ$ C für 10-12 Minuten} Ober- und Unterhitze backen.

\end{multicols}
\vfill
\end{document}