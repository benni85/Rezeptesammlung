\documentclass[a6paper, twoside]{report}
\usepackage[a6paper, top=7mm, inner=10mm, right=10mm, bottom=10mm, landscape]{geometry}
\usepackage[ngerman]{babel}
\usepackage[utf8]{inputenc}
\usepackage{multicol}
\pagestyle{plain}



\parindent0pt	

\pagestyle{empty}
\begin{document}

\textbf{{\LARGE Pesto Verde}}%%%%%%%%%%%%%%%%%%%%%%%%%%%%%%%%%%%%%%%%%%%%%%%%%%%%%%%%%%%%%%%%%%%%%%%%


\hrulefill
\vspace*{\fill}
\begin{multicols}{2}	


\begin{itemize}
\item 75 g 	frischer Basilikum 
\item 2 EL 	Pinienkerne
\item 1 Zehe 	Knoblauch
\item $\frac{1}{2}$ TL 	Salz
\item 200 ml 	Olivenöl, extra vergine
\item 2 EL 	frisch geriebener Parmesan

\end{itemize}

\end{multicols}

\vspace{1cm}
\begin{center}
[Zutaten für ca. 300 ml]
\end{center}

\vfill
\newpage
\textbf{{\LARGE Zubereitung}}%%%%%%%%%%%%%%%%%%%%%%%%%%%%%%%%%%%%%%%%%%%%%%%%%%%%%%%%%%%%%%%%%%%%%%%%

\hrulefill

\vspace*{\fill}
\begin{multicols}{2}
\columnseprule1pt

Alle Zutaten bis auf den Parmesankäse mit einem Pürierstab mixen, dabei die einzelnen Zutaten nach und nach zugeben. 
Wenn alles püriert ist, den Parmesan unterrühren.\\
Alternativ kann auch es auch schonender in einem Mörser zubereitet werden.\\

Das Pesto in ein Glas füllen und gut verschließen.
Das Pesto hält sich im Kühlschrank einige Wochen.\\

Bevor das Pesto mit der abgekochten Pasta vermischt wird nocheinmal kurz aufkochen und einen Schuss Sahne zugeben. 

\end{multicols}
\vfill
\end{document}