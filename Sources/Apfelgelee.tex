\documentclass[a6paper, twoside]{report}
\usepackage[a6paper, top=7mm, inner=10mm, right=10mm, bottom=10mm, landscape]{geometry}
\usepackage[ngerman]{babel}
\usepackage[utf8]{inputenc}
\usepackage{multicol}
\pagestyle{plain}




\parindent0pt	

\pagestyle{empty}
\begin{document}

\textbf{{\LARGE gelée de pommes}}%%%%%%%%%%%%%%%%%%%%%%%%%%%%%%%%%%%%%%%%%%TITEL
%%%%%%%%%%%%%%%%%%%%%%%%%%%%%%%%%%%%%%%%%%%%%%%%%%%%%%%%%%%%%UNTERTITEL
%	https://www.kuechengoetter.de/rezepte/apfelgelee-37334	%%QUELLE als Kommentar

\hrulefill
%\vspace*{\fill}

\vspace{0.5cm}

\begin{multicols}{2}
\textbf{ingredients}
\begin{itemize}

\item pommes 
\item confiture sucre
\item jus de citron

\vspace*{\fill}
\end{itemize}
\columnbreak
Ma mère utilise toujours des "Korn-äpfel", (pommes de maïs). Mais en fait seulement parce qu'ils poussent sur notre propre arbre.
Vous pouvez utiliser toutes les pommes ou simplement sauter les premières étapes et utiliser du jus de pomme. 
%Meine Mutter verwendet immer "Kornäpfel", (Kornäpfel). Aber eigentlich nur weil die an unserem eigenen Baum wachsen.
%Man kann grundsätzlich alle Äpfel verwenden, oder einfach die ersten Schritte überspringen und Apfelsaft benutzen. 
\end{multicols}

\vspace{0.5cm}
\textbf{verrerie}
\begin{multicols}{2}

Les verres et couvercles lavés doivent être immergés dans de l'eau bouillante avant le remplissage.
%Die ausgewaschenen Gläser und Deckel müssen vor dem Befüllen in kochendes Wasser getaucht werden.
\columnbreak

Après le remplissage, refermer immédia-tement et placer sur un torchon de cuisine avec le couvercle vers le bas. Après refroidissement, le couvercle doit se plier légèrement vers l'intérieur.
%Nach dem Befüllen sofort verschließen und mit dem Deckel nach unten auf ein Küchentuch stellen. Nach dem Abkühlen
%muss sich der Deckel leicht nach innnen biegen.

\end{multicols}
\vfill									%%Neue//

%\begin{center}
%%
%[Ingrédients pour 4 personnes]%%%%%%%%%%%%%%%%%%%%%%%%%%%%%%%%%%%%%%%%%%%%%%%MENGE\end{center}


\vfill
\newpage
\textbf{{\LARGE préparation}}%%%%%%%%%%%%%%%%%%%%%%%%%%%%%%%%%%%%%%%%%%%%%%%%

\hrulefill

\vspace*{\fill}
\begin{multicols}{2}
\columnseprule1pt

Lavez et séchez les pommes et enlevez généreusement tous les points noirs. Couper les pommes en gros morceaux.\newline 
Mettez-les dans une casserole avec un peu d'eau et portez à ébullition avec le couvercle fermé.\newline

 Dès que l'eau bout, réduire un peu le feu et laisser bouillir les pommes pendant environ 40 minutes jusqu'à ce qu'elles se désintègrent.\newline
 
 Remuer de temps en temps. Tapisser un grand tamis d'un linge qui passe et le suspendre au-dessus d'un bol. Ajouter le mélange de pommes et égoutter pendant plusieurs heures ou toute la nuit. \newline 

Ne pas presser, sinon le jus risque de devenir trouble.\newline


 Mesurer le jus de pomme obtenu. Porter à ébullition dans une casserole avec la même quantité de sucre et de jus de citron et cuire à feu doux pendant environ 30 minutes. \newline
 
%Il est absolument nécessaire de faire un test de gélification : Déposer une goutte de gelée liquide sur une petite assiette. S'il se gélifie, remplissez la gelée. Sinon, réduisez encore plus.


%    Die Äpfel waschen, trocknen und alle schlechten Stellen großzügig herausschneiden. Äpfel in grobe Stücke schneiden. Mit 500 ml Wasser in einen Topf geben und mit geschlossenem Deckel aufkochen. Sobald das Wasser kocht, die Hitze etwas reduzieren und die Äpfel ca. 40 Min. kochen lassen, bis sie zerfallen sind. Dabei ab und zu umrühren. Ein großes Sieb mit einem Passiertuch oder einer Mullwindel auslegen und über eine Schüssel hängen. Die Apfelmasse hineingeben und mehrere Stunden oder über Nacht abtropfen lassen. Auf keinen Fall ausdrücken, sonst wird der Saft trübe.
   % Den gewonnenen Apfelsaft abmessen. In einem Topf mit der gleichen Menge Zucker (auf 1 l Saft 1 kg Zucker nehmen) und dem Zitronensaft aufkochen und bei kleiner Hitze ca. 30 Min. kochen lassen. 
  % Unbedingt eine Gelierprobe machen: Einen Tropfen flüssiges Gelee auf einen kleinen Teller geben. Geliert er, das Gelee abfüllen. Sonst noch weiter einkochen.



\end{multicols}
\vfill
\end{document}