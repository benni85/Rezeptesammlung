\documentclass[a6paper, twoside]{report}
\usepackage[a6paper, top=7mm, inner=10mm, right=10mm, bottom=10mm, landscape]{geometry}
\usepackage[ngerman]{babel}
\usepackage[utf8]{inputenc}
\usepackage{multicol}
\pagestyle{plain}





\parindent0pt	

\pagestyle{empty}
\begin{document}

\textbf{{\LARGE Marokkanisches Couscous}}%%%%%%%%%%%%%%%%%%%%%%%%%%%%%%%%%%%%%%%%%%TITEL
%Joghurt-Drink%%%%%%%%%%%%%%%%%%%%%%%%%%%%%%%%%%%%%%%%%%%%%%%%%%%%%%%%%%%%UNTERTITEL


\hrulefill
\vspace*{\fill}
\begin{multicols}{2}	


\begin{itemize}
\item  $1\frac{1}{2}$ Tassen Couscous
\item  $\frac{1}{4}$ TL Salz
\item  $\frac{1}{2}$ TL  Zimt
\item $\frac{1}{2}$ TL Paprika
\item  $\frac{1}{8}$ TL gemahlener Kümmel
\item  $1\frac{1}{2}$ Tassen kochendes Wasser
\item $1\frac{1}{2}$ EL Olivenöl
\item $2\frac{1}{2}$ EL Essig
\item 2-3 Tassen Blattspinat
\item 1 Tasse gehackte Orangenspalten
\end{itemize}
\vfill									
\end{multicols}

\vspace{2cm}			%
\begin{center}			%
Als Beilage für 2-4 Personen
\end{center}


\vfill
\newpage
\textbf{{\LARGE Zubereitung}}%%%%%%%%%%%%%%%%%%%%%%%%%%%%%%%%%%%%%%%%%%%%%%%%

\hrulefill

\vspace*{\fill}
\begin{multicols}{2}
\columnseprule1pt

Couscous, Salz und Gewürze mischen und in einem festschließendem Topf mit dem kochenden Wasser
gut mischen.\newline 

Für mindestens 8 Minuten ruhen lassen bis das ganze Wasser aufgesogen ist.\newline

Öl und Essig in einer Tasse mischen und wenn der Couscous fertig ist unterrühren, dann abschmecken.

Den Blattspinat und die Orangenspalten unterheben.\\

Kalt oder bei Raumtemperatur servieren.

\end{multicols}
\vfill
\end{document}