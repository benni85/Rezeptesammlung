\documentclass[a6paper, twoside]{report}
\usepackage[a6paper, top=7mm, inner=10mm, right=10mm, bottom=10mm, landscape]{geometry}
\usepackage[ngerman]{babel}
\usepackage[utf8]{inputenc}
\usepackage{multicol}
\pagestyle{plain}




\parindent0pt	

\pagestyle{empty}
\begin{document}

\textbf{{\LARGE Cantuccini}}%%%%%%%%%%%%%%%%%%%%%%%%%%%%%%%%%%%%%%%%%%TITEL
%%%%%%%%%%%%%%%%%%%%%%%%%%%%%%%%%%%%%%%%%%%%%%%%%%%%%%%%%%%%%UNTERTITEL
%Von Steffi			%%QUELLE als Kommentar

\hrulefill
\vspace*{\fill}
\begin{multicols}{2}	


\begin{itemize}
\item 175 g 	Mandeln
\item 250 g 	Mehl
\item 180 g 	Zucker
\item 1 TL 	Backpulver
\item 2 Pk 	Vanillezucker
\item $\frac{1}{2}$ Flasche Bittermandelaroma
\item 1 Prise	Salz
\item 25 g	Butter
\item 2 	Eier
\end{itemize}
\end{multicols}
\vfill									%%Neue//




\vfill
\newpage
\textbf{{\LARGE Zubereitung}}%%%%%%%%%%%%%%%%%%%%%%%%%%%%%%%%%%%%%%%%%%%%%%%%

\hrulefill

\vspace*{\fill}
\begin{multicols}{2}
\columnseprule1pt

Die Mandeln kurz in kochendes Wasser tauchen, in ein Sieb kippen, kalt abbrausen und häuten.
Auf einem Tuch über Nacht trocknen lassen.\newline

Mehl, Zucker, Backpulver, Vanillezucker, Aroma und Salz in eine Schüssel
geben. Butter und Eier hinzugeben.\newline

Alle Zutaten zu einem klebrigen Knetteig verarbeiten. Die Mandeln
unterkneten.
Den Teig mit etwas Mehl zu einer Kugel formen und für 30 Minuten kaltstellen.\newline

Anschliessend den Teig in 6 Teile schneiden und aus jedem
eine ca 25 cm lange Rolle formen.
Die Rollen auf einem mit Backpapier belegtem Blech in 8 cm Abstand
aufreihen.\newline

Im Ofen bei \textbf{200$^\circ$ C für 15 Minuten} vorbacken, kalt werden lassen und
schräg in 1 cm dicke Scheiben schneiden.
Dann für \textbf{10 Minuten} im Ofen rösten.
Die Cantuccini müssen zum Schluss goldbraun sein.
In einem geschlossenen Behälter aufbewahren.

\end{multicols}
\vfill
\end{document}