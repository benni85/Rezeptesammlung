\documentclass[a6paper, twoside]{report}
\usepackage[a6paper, top=7mm, inner=10mm, right=10mm, bottom=10mm, landscape]{geometry}
\usepackage[ngerman]{babel}
\usepackage[utf8]{inputenc}
\usepackage{multicol}
\pagestyle{plain}




\parindent0pt	

\pagestyle{empty}
\begin{document}

\textbf{{\LARGE Ravioli}}%%%%%%%%%%%%%%%%%%%%%%%%%%%%%%%%%%%%%%%%%%TITEL
%%%%%%%%%%%%%%%%%%%%%%%%%%%%%%%%%%%%%%%%%%%%%%%%%%%%%%%%%%%%%UNTERTITEL
%		%%QUELLE als Kommentar

\hrulefill
\vspace*{\fill}
\begin{multicols}{2}	


\begin{itemize}
\item 100 g Weizenmehl
\item 100 g Hartweizengrieß
\item 1 Ei
\item Etwas Safran, gemahlen
\item 100 g Ziegenfrischkäse
\item 100 g Ricotta
\item 30 g Pinienkerne
\item 5 getrocknete Tomaten in Öl
\item etwas Butter
\item Salz und Pfeffer
\end{itemize}
\end{multicols}
\vfill									%%Neue//

\vspace{0.5cm}
%
\begin{center}
%
[Zutaten für 2 Portionen]%%%%%%%%%%%%%%%%%%%%%%%%%%%%%%%%%%%%%%%%%%%%%%%MENGE
\end{center}


\vfill
\newpage
\textbf{{\LARGE Zubereitung}}%%%%%%%%%%%%%%%%%%%%%%%%%%%%%%%%%%%%%%%%%%%%%%%%

\hrulefill

\vspace*{\fill}
\begin{multicols}{2}
\columnseprule1pt

Aus dem Mehl, Grieß, dem Ei, 50 ml Wasser, Safran und einer Messerspitze Salz einen geschmeidigen, nicht klebenden Teig kneten.\newline

Die Pinienkerne in einer Pfanne rösten.\newline

Für die Füllung den Ricotta, Frischkäse, die abgetropften Tomaten und die Pinienkerne fein pürieren.\newline

Den Teig maximal 3 mm dick ausrollen und mit einem Trinkglas ausstechen.\newline
In die Mitte der runden Teiglinge jeweils einen Teelöffel Füllung geben. Die Teiglinge dann in der Mitte umklappen und mit einer Gabel am Rand festdrücken.\newline

Sollten die Ravioli nicht sofort gekocht werden, die fertigen Nudeln bis zur Verwendung auf ein mit Mehl bestreutem Backblech legen.\newline

Wasser in einem Topf aufkochen, die Temperatur herunterdrehen bis das Wasser nur noch simmert. Die Ravioli für 3-5 Minuten in das heisse Wasser geben.\newline

Zum Servieren die Ravioli mit etwas geschmolzener Butter übergießen. 

\end{multicols}
\vfill
\end{document}