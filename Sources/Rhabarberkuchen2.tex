\documentclass[a6paper, twoside]{report}
\usepackage[a6paper, top=7mm, inner=10mm, right=10mm, bottom=10mm, landscape]{geometry}
\usepackage[ngerman]{babel}
\usepackage[utf8]{inputenc}
\usepackage{multicol}
\pagestyle{plain}



\parindent0pt	

\pagestyle{empty}
\begin{document}

\textbf{{\LARGE Rhabarberkuchen}}%%%%%%%%%%%%%%%%%%%%%%%%%%%%%%%%%%%%%%%%%%%%%%%%%%%%%%%%%%%%%%%%%%%%%%%%


\hrulefill
\vspace*{\fill}
\begin{multicols}{2}	



Belag:
\begin{itemize}
\item 500 g Rhabarber
\item 90 g Zucker
\end{itemize}
Teig:
\begin{itemize}
\item 80 g Butter
\item 160 g Zucker
\item 2 Eier
\item 80 g Mehl
\item 100 g Mondamin (Stärkemehl aus Mais)
\item 3 TL Backpulver
\item $\frac{1}{2}$ Tasse Milch
\end{itemize}

\end{multicols}
\vfill
\newpage
\textbf{{\LARGE Zubereitung}}%%%%%%%%%%%%%%%%%%%%%%%%%%%%%%%%%%%%%%%%%%%%%%%%%%%%%%%%%%%%%%%%%%%%%%%%

\hrulefill

\vspace*{\fill}
\begin{multicols}{2}
\columnseprule1pt

Geschnittener Rhabarber in einer Springform auf dem Backpapier verteilen und den 90 g Zucker bestreuen.
Butter schaumig rühren, Zucker und Eier abwechselnd zugeben, rühren bis der Zucker sich aufgelöst hat.
Mehl, Backpuölver und Mondamin mischen und einrühren.\\

Milch zugeben bis der Teig dickflüssig ist (löst sich schwer reissend von den Rührern).\\

Den Teig über den Rhabarber geben und \textbf{40-50 Min bei mittlerer Hitze backen}.\\

Nach dem Backen abkühlen lassen, Springform lösen und auf Tortenform geben.



\end{multicols}
\vfill
\end{document}