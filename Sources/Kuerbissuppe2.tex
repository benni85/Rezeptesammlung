\documentclass[a6paper, twoside]{report}
\usepackage[a6paper, top=7mm, inner=10mm, right=10mm, bottom=10mm, landscape]{geometry}
\usepackage[ngerman]{babel}
\usepackage[utf8]{inputenc}
\usepackage{multicol}
\pagestyle{plain}



\parindent0pt	

\pagestyle{empty}
\begin{document}

\textbf{{\LARGE Kürbissuppe}}%%%%%%%%%%%%%%%%%%%%%%%%%%%%%%%%%%%%%%%%%%TITEL
%%%%%%%%%%%%%%%%%%%%%%%%%%%%%%%%%%%%%%%%%%%%%%%%%%%%%%%%%%%%UNTERTITEL
%Aus Kochbuch: Blablub Seite 34 			%%QUELLE als Kommentar

\hrulefill

Ein Rezept von Oma Lisl
\vspace*{\fill}
\begin{multicols}{2}	


\begin{itemize}
\item 1 Kürbis
\item 1 Zwiebel
\item 1 EL Butter
\item 1 Scheibe Ingwer
\item 1 Becher Sahne
\end{itemize}
\vfill									%%Neue//
\columnbreak								%%Spalte

\begin{itemize}
\item 1 Lorbeerblatt
\item Gemüsebrühe
\item Salz
\item Pfeffer 
\end{itemize}
\end{multicols}

\vspace{2cm}			%


\vfill
\newpage
\textbf{{\LARGE Zubereitung}}%%%%%%%%%%%%%%%%%%%%%%%%%%%%%%%%%%%%%%%%%%%%%%%%

\hrulefill

\vspace*{\fill}
\begin{multicols}{2}
\columnseprule1pt

Den Kürbis und die Kartoffel in Würfel schneiden und mit dem Ingwer und der gehackten Zwiebel in
Butter andünsten.

Mit der Gemüsebrühe aufgießen und köcheln lassen bis der Kürbis weich genug ist um
 alles zu pürieren.

Dann mit der Sahne verfeinern und mit Salz und Pfeffer abschmecken.

%Rezept von Oma Lisl

\end{multicols}
\vfill
\end{document}