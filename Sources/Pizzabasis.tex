\documentclass[a6paper, twoside]{report}
\usepackage[a6paper, top=7mm, inner=10mm, right=10mm, bottom=10mm, landscape]{geometry}
\usepackage[ngerman]{babel}
\usepackage[utf8]{inputenc}
\usepackage{multicol}
\pagestyle{plain}



\parindent0pt	

\pagestyle{empty}
\begin{document}

\textbf{{\LARGE Pizza}}

\hrulefill
\vspace*{\fill}
\begin{multicols}{2}	

Für den Teig:
\begin{itemize}
\item 250 ml Wasser
\item 460 g Mehl
\item 20 g Hefe
\item 10 g Salz
\item 12 ml Olivenöl
\item  1 Prise Zucker
\end{itemize}

Tomatensauce:
\begin{itemize}
\item 250 ml passierte Tomaten
\item 2 EL Tomatenmark
\item 1 kleine Zwiebel
\item 1 Knoblauchzehe
\item Basilikum, Rosmarin, Oregan
\item Olivenöl, kaltgepresst
\item Zucker
\item grobkörniges Meersalz
\item Pfeffer
\end{itemize}
\vfill									
\end{multicols}

\vspace{0.5 cm}			%
\begin{center}			%
[Diese Menge reicht für 2 Bleche]
\end{center}


\vfill
\newpage
\textbf{{\LARGE Zubereitung}}%%%%%%%%%%%%%%%%%%%%%%%%%%%%%%%%%%%%%%%%%%%%%%%%

\hrulefill

\vspace*{\fill}
\begin{multicols}{2}
\columnseprule1pt

\textbf{Pizzaboden}\newline
Die Hefe in das Wasser, Öl, Salz und Zucker verrühren. Nach und nach das Mehl zugeben und
zu einem glatten Teig kneten.\newline
$\frac{1}{2}$h an einem warmen Ort gehen lassen, dann erneut zusammenkneten und im 
Kühlschrank für zwei Tage ruhen lassen. \newline
Der Teig darf nicht trocknen.\newline
Vor der Verwendung den Teig auf Raumtemperatur aufwärmen lassen.
Dünn auswälzen und vor dem Belegen mit Olivenöl bestreichen.\newline Bei maximaler Temperatur
\textbf{für 5 - 10 Minuten} backen.\newline
\columnbreak

\textbf{Tomatensauce}\newline
Die Zwiebel und den Knoblauch feinhacken und im heißen Öl glasig andünsten. Die passierten Tomaten dazugeben
und aufkochen.\newline
Die restlichen Zutaten zugeben und \textbf{für 20 - 25 Minuten} leicht köcheln lassen.



\end{multicols}
\vfill
\end{document}