\documentclass[a6paper, twoside]{report}
\usepackage[a6paper, top=7mm, inner=10mm, right=10mm, bottom=10mm, landscape]{geometry}
\usepackage[ngerman]{babel}
\usepackage[utf8]{inputenc}
\usepackage{multicol}
\pagestyle{plain}




\parindent0pt	

\pagestyle{empty}
\begin{document}

\textbf{{\LARGE Osterlamm}}%%%%%%%%%%%%%%%%%%%%%%%%%%%%%%%%%%%%%%%%%%TITEL
%%%%%%%%%%%%%%%%%%%%%%%%%%%%%%%%%%%%%%%%%%%%%%%%%%%%%%%%%%%%%UNTERTITEL
%		%%QUELLE als Kommentar

\hrulefill
\vspace*{\fill}
\begin{multicols}{2}	


\begin{itemize}
\item 75 g Butter
\item 100 g Zucker
\item 1 Pk Vanillinzucker	
\item 2 Eier
\item 10 Tropfen Rum-Aroma
\item 125 g Weizenmehl
\item 1 gestr. TL Backpulver
\item 1 Prise Salz
\end{itemize}
\end{multicols}
\vfill									%%Neue//

\vspace{2cm}
%
\begin{center}
%

\end{center}


\vfill
\newpage
\textbf{{\LARGE Zubereitung}}%%%%%%%%%%%%%%%%%%%%%%%%%%%%%%%%%%%%%%%%%%%%%%%%

\hrulefill

\vspace*{\fill}
\begin{multicols}{2}
\columnseprule1pt

Das Fett schaumig rühren und nach und nach alle Zutaten bis auf das Mehl und Backpulver zugeben.

Das gesiebte Mehl und Backpulver esslöffelweise unterrühren.\newline

Den Teig dann in die gefettete Form geben und bei \textbf{ 180$^\circ$ C für 35 bis 45 Minuten backen}.\newline

Das Lamm etwa 5 Minuten in der Form abkühlen lassen, dann herausnehmen und erkalten lassen.


\end{multicols}
\vfill
\end{document}