\documentclass[a6paper, twoside]{report}
\usepackage[a6paper, top=7mm, inner=10mm, right=10mm, bottom=10mm, landscape]{geometry}
\usepackage[ngerman]{babel}
\usepackage[utf8]{inputenc}
\usepackage{multicol}
\pagestyle{plain}



\parindent0pt	

\pagestyle{empty}
\begin{document}

\textbf{{\LARGE Gazpacho}}%%%%%%%%%%%%%%%%%%%%%%%%%%%%%%%%%%%%%%%%%%%%%%%%%%%%%%%%%%%%%%%%%%%%%%%%


\hrulefill
\vspace*{\fill}
\begin{multicols}{2}	


\begin{itemize}
\item $\frac{1}{2}$ kg 	Kirschtomaten
\item $\frac{1}{2}$ 	Salatgurke
\item 1  	grüne Paprikaschote, geschält und entkernt
\item 1  	rote Paprikaschote, geschält und entkernt
\item 2 Zehen 	Knoblauch, zerdrückt
\item 1  	Zwiebel (Gemüsezwiebel)
\item 750 ml 	passierte Tomate 
\item 250 ml 	kalte Gemüsebrühe
\item 75 ml 	Öl (Olivenöl)
\item 50 ml 	weisser Balsamico
\item Salz und Pfeffer
\item 1 kleine 	Chilischote, ohne Kerne
\item Zucker 
\end{itemize}

\end{multicols}
\vfill
\newpage
\textbf{{\LARGE Zubereitung}}%%%%%%%%%%%%%%%%%%%%%%%%%%%%%%%%%%%%%%%%%%%%%%%%%%%%%%%%%%%%%%%%%%%%%%%%

\hrulefill

\vspace*{\fill}
\begin{multicols}{2}
\columnseprule1pt



Das Gemüse in feine Würfel schneiden und in eine Schüssel geben. 
Tomatenpüree mit Gemüsebrühe, Essig, Öl, Knoblauch und Chili vermischen und über das Gemüse geben. 
Mit Salz, Pfeffer und Zucker abschmecken und richtig kalt (auf Eis) servieren.\\

Dazu passt geröstetes Bauernbrot, mit Knoblauch abgerieben, leicht gesalzen und mit Olivenöl beträufelt. 


\end{multicols}
\vfill
\end{document}