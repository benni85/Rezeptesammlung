\documentclass[a6paper, twoside]{report}
\usepackage[a6paper, top=7mm, inner=10mm, right=10mm, bottom=10mm, landscape]{geometry}
\usepackage[ngerman]{babel}
\usepackage[utf8]{inputenc}
\usepackage{multicol}
\pagestyle{plain}




\parindent0pt	

\pagestyle{empty}
\begin{document}

\textbf{{\LARGE Palatschinken}}%%%%%%%%%%%%%%%%%%%%%%%%%%%%%%%%%%%%%%%%%%TITEL
%%%%%%%%%%%%%%%%%%%%%%%%%%%%%%%%%%%%%%%%%%%%%%%%%%%%%%%%%%%%%UNTERTITEL
%Von Steffi			%%QUELLE als Kommentar

\hrulefill
\vspace*{\fill}
\begin{multicols}{2}	


\begin{itemize}
\item 2 Eier
\item 550 ml Milch
\item 300 g Mehl
\item 1 MSp Salz
\item Fett für die Pfanne
\end{itemize}
\end{multicols}
\vfill									




\vfill
\newpage
\textbf{{\LARGE Zubereitung}}%%%%%%%%%%%%%%%%%%%%%%%%%%%%%%%%%%%%%%%%%%%%%%%%

\hrulefill

\vspace*{\fill}
\begin{multicols}{2}
\columnseprule1pt

Aus den Zutaten einen glatten Teig rühren. Diesen dann 15-30 Minuten 
im Kühlschrank zugedeckt ruhen lassen.\newline

Während das Fett in der Pfanne erhitzt wird, den Teig nochmals durchrühren.\newline

Mit einem Schöpflöffel portionsweise Teig in die Pfanne gießen. Die Unterseite
stocken und goldbraun backen lassen, dann wenden bis diese Seite ebenfalls 
goldbraun ist.\newline

Gegebenenfalls die fertigen Palatschinken im Ofen bei 100$^\circ$ C warmstellen bis
alle fertig sind.

Der Teig eignet sich für herzhafte oder süße Beläge.
Der ungebackene Teig kann im Kühlschrank bis zu drei Tagen abgedeckt aufbewahrt werden.

\end{multicols}
\vfill
\end{document}