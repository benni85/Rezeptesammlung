\documentclass[a6paper, twoside]{report}
\usepackage[a6paper, top=7mm, inner=10mm, right=10mm, bottom=10mm, landscape]{geometry}
\usepackage[ngerman]{babel}
\usepackage[utf8]{inputenc}
\usepackage{multicol}
\pagestyle{plain}




\parindent0pt	

\pagestyle{empty}
\begin{document}

\textbf{{\LARGE Marinara-Sauce}}%%%%%%%%%%%%%%%%%%%%%%%%%%%%%%%%%%%%%%%%%%TITEL
%%%%%%%%%%%%%%%%%%%%%%%%%%%%%%%%%%%%%%%%%%%%%%%%%%%%%%%%%%%%%UNTERTITEL
%Von Steffi			%%QUELLE als Kommentar

\hrulefill
\vspace*{\fill}
\begin{multicols}{2}	


\begin{itemize}
\item $\frac{1}{2}$ Tasse Olivenöl
\item 2 Zwiebeln
\item 2 Knoblauchzehen
\item 2 Selleriestangen
\item 2 geschälte Karotten
\item $\frac{1}{2}$ TL Meersalz
\item $\frac{1}{2}$ TL schwarzer Pfeffer
\item 2 Dosen geschälte Tomaten (900 g)
\item 2 getrocknete Lorbeerblätter
\end{itemize}
\end{multicols}
\vfill									%%Neue//


\begin{center}			%
Ausreichend für 8 Tassen
\end{center}

\vfill
\newpage
\textbf{{\LARGE Zubereitung}}%%%%%%%%%%%%%%%%%%%%%%%%%%%%%%%%%%%%%%%%%%%%%%%%

\hrulefill

\vspace*{\fill}
\begin{multicols}{2}
\columnseprule1pt

In einem großen Top das Öl über mittlerer Hitze erwärmen.
Die Zwiebeln und den Knoblauch sautieren bis sie durchsichtig werden (ca. 10 Min).\newline

Den Sellerie und die Karotten schneiden und mit Salz und Pfeffer
zugeben.\newline

Das Ganze für ca. 10 Minuten sautieren bis das Gemüse weich ist.
Dann die Tomaten und Lorbeerblätter zugeben un offen über niedriger
Hitze köcheln lassen bis die Sauce eindickt (ca. 1h).\newline

Zum Schluss die Lorbeerblätter herausnehmen und je nach
Geschmack mit zusätzlich Salz und Pfeffer würzen.

\end{multicols}
\vfill
\end{document}