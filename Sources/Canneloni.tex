\documentclass[a6paper, twoside]{report}
\usepackage[a6paper, top=7mm, inner=10mm, right=10mm, bottom=10mm, landscape]{geometry}
\usepackage[ngerman]{babel}
\usepackage[utf8]{inputenc}
\usepackage{multicol}
\pagestyle{plain}




\parindent0pt	

\pagestyle{empty}
\begin{document}

\textbf{{\LARGE Cannelloni}}%%%%%%%%%%%%%%%%%%%%%%%%%%%%%%%%%%%%%%%%%%TITEL
%%%%%%%%%%%%%%%%%%%%%%%%%%%%%%%%%%%%%%%%%%%%%%%%%%%%%%%%%%%%%UNTERTITEL
%		%%QUELLE als Kommentar

\hrulefill
\vspace*{\fill}
\begin{multicols}{2}	


\begin{itemize}
\item $\frac{1}{2}$ Paket Cannelloni
\item 300 g stückige Tomaten
\item 225 g Blattspinat, tiefgefroren
\item 2 Dosen	stückige Tomaten
\item 125 g Ricotta
\item 10 g Pinienkerne, geröstet
\item 1 Ei
\item 1 Schalotte
\item 5 EL Parmesan, gerieben
\item 20 g Mehl
\item 20 g Butter
\item 250 ml Milch
\item 1 Zehe Knoblauch, gepresst
\item Salz, Pfeffer, Muskat, Oregano
\end{itemize}
\end{multicols}
\vfill									%%Neue//

\vspace{1cm}
%
\begin{center}
%
[Zutaten für 2 Portionen]%%%%%%%%%%%%%%%%%%%%%%%%%%%%%%%%%%%%%%%%%%%%%%%MENGE
\end{center}


\vfill
\newpage
\textbf{{\LARGE Zubereitung}}%%%%%%%%%%%%%%%%%%%%%%%%%%%%%%%%%%%%%%%%%%%%%%%%

\hrulefill

\vspace*{\fill}
\begin{multicols}{2}
\columnseprule1pt
Den Ricotta mit dem Ei und zwei EL Käse mischen und mit Salz und Pfeffer abschmecken, die Pinienkerne hinzugeben.\newline
Die Zwiebel kleinschneiden und mit etwas Öl anschwitzen, die Knoblauchzehe kurz mitbraten. Zwiebeln und Knoblauch dann auf zwei Töpfe mit dem Spinat und den Tomaten aufteilen.\newline
Beides solange kochen lassen bis die Flüssigkeit größtenteils verdmpft ist, dann mit Salz, Pfeffer und Muskat abschmecken.\newline
%Den Spinat solange Kochen lassen bis die Flüssigkeit verdampft ist. Dann mit Salz, Pfeffer und Muskat abschmecken und abkühlen lassen.\newline
%Die Tomaten einkochen lassen, und ebenfalls mit den Gewürzen abschmecken.\newline

Aus Butter, Mehl und Milch eine Béchamelsauce herstellen, einen EL Parmesan unterheben und mit Salz, Pfefer und Muskat abschmecken.\newline

%Die Butter in einem Topf zerlassen und das Mehl unterrühren. Unter Rühren bei mittlerer Hitze zwei Minuten anschwitzen, dann unter Rühren mit dem Schneebesen mit der Milch ablöschen und zwei Minuten sanft köcheln lassen. Zwei EL geriebenen Parmesan unterheben und die Sauce mit Salz, Pfeffer und Muskat abschmecken.\newline
Eine Form einfetten und den Boden mit der Hälfte der Tomatensauce bedecken. Den Blattspinat mit dem Ricotta vermengen und in die Cannelloni füllen, dann diese in die Form schlichten. Den Rest der Tomatensauce auf den Cannelloni verteilen. Danach die Béchamelsauce darüber löffeln und mit dem restlichen Käse bestreuen. \newline

Mit Alufolie abdecken und bei\textbf{ 225$\circ$C Ober-/Unterhitze 15 min.} backen. Die Alufolie abnehmen und weitere 10 min. backen, bis eine schöne Bräunung entsteht. 


\end{multicols}
\vfill
\end{document}