\documentclass[a6paper, twoside]{report}
\usepackage[a6paper, top=7mm, inner=10mm, right=10mm, bottom=10mm, landscape]{geometry}
\usepackage[ngerman]{babel}
\usepackage[utf8]{inputenc}
\usepackage{multicol}
\pagestyle{plain}



\parindent0pt	

\pagestyle{empty}
\begin{document}

\textbf{{\LARGE Muesli}}%%%%%%%%%%%%%%%%%%%%%%%%%%%%%%%%%%%%%%%%%%%%%%%%%%%%%%%%%%%%%%%%%%%%%%%%


\hrulefill
\vspace*{\fill}
\begin{multicols}{2}	


\begin{itemize}
\item 100 g Haferflocken
\item 50 g Rosinen
\item 250 ml Milch
\item 60 ml Sahne
\item 2 Äpfel 
\item 1 Banane
\item $\frac{1}{2}$ Zitrone
\item 30 g gehackte Haselnüsse
\item 2 EL Honig
\item 1 TL Zimt
\end{itemize}

\end{multicols}
\vfill
\newpage
\textbf{{\LARGE Zubereitung}}%%%%%%%%%%%%%%%%%%%%%%%%%%%%%%%%%%%%%%%%%%%%%%%%%%%%%%%%%%%%%%%%%%%%%%%%

\hrulefill

\vspace*{\fill}
\begin{multicols}{2}
\columnseprule1pt

Die Haferflocken mit den Rosinen über Nacht in Milch und Sahne einweichen (im Kühlschrank).\\
Die Äpfel raspeln, die Banane zerdrücken und den Saft der Zitrone untermischen um zu Verhindern
das das Obst braun wird. Dann alle anderen Zutaten zugeben und die 
Hafenflockenmasse unterheben.\\

Bis zum Servieren im Kühlschrank aufbewahren.




\end{multicols}
\vfill
\end{document}