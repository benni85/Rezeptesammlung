\documentclass[a6paper, twoside]{report}
\usepackage[a6paper, top=7mm, inner=10mm, right=10mm, bottom=10mm, landscape]{geometry}
\usepackage[ngerman]{babel}
\usepackage[utf8]{inputenc}
\usepackage{multicol}
\pagestyle{plain}



\parindent0pt	

\pagestyle{empty}
\begin{document}

\textbf{{\LARGE Mezzani ala Pizzaiola}}%%%%%%%%%%%%%%%%%%%%%%%%%%%%%%%%%%%%%%%%%%%%%%%%%%%%%%%%%%%%%%%%%%%%%%%%


\hrulefill

Nudeln nach Art der Pizzabäcker\\
\vspace*{\fill}
\begin{multicols}{2}	


\begin{itemize}
\item 400 g Mezzani oder Makkaroni
\item 400 g Rinderfilet in Scheiben von ca 5 mm
\item 1 Knoblauchzehe
\item $\frac{1}{2}$ Zwiebel
\item 30 g Kapern
\item 3 EL Olivenöl
\item 1 TL Salz
\item schwarzer Pfeffer
\item 3 EL Petersilie gehackt
\item 2 TL Oregano gehackt
\item 500 g gehäutete Tomaten
\item frisch geriebener Parmesan
\end{itemize}


\end{multicols}
\vspace{1cm}
\begin{center}
[Zutaten für 4 Personen]
\end{center}

\vfill
\newpage
\textbf{{\LARGE Zubereitung}}%%%%%%%%%%%%%%%%%%%%%%%%%%%%%%%%%%%%%%%%%%%%%%%%%%%%%%%%%%%%%%%%%%%%%%%%

\hrulefill

\vspace*{\fill}
\begin{multicols}{2}
\columnseprule1pt


Die Nudeln in Salzwasser al dente kochen und abtropfen lassen.\\

Den Knoblauch und die Zwiebel schälen und kleinschneiden. Das Fruchtfleisch der Tomaten etwas zerkleinern.\\

Das Olivenöl in einer beschichteten Pfanne erhitzen und die Filetscheiben bei hoher Hitze kurz von beiden Seiten anbraten. Anschließend das Fleisch aus der Pfanne nehmen.
Knoblauch und Zwiebel in der Pfanne anbraten, die Gewürze, Kräuter und die Tomaten zugeben und erhitzen.\\

Das Fleisch zugeben und ca 6 Minuten bei mittlerer Hitze schmoren lassen.\\
 
Die Soße zu den Nudeln servieren.




\end{multicols}
\vfill
\end{document}