\documentclass[a6paper, twoside]{report}
\usepackage[a6paper, top=7mm, inner=10mm, right=10mm, bottom=10mm, landscape]{geometry}
\usepackage[ngerman]{babel}
\usepackage[utf8]{inputenc}
\usepackage{multicol}
\pagestyle{plain}




\parindent0pt	

\pagestyle{empty}
\begin{document}

\textbf{{\LARGE Heffnkliss mit Vanillesauce}}%%%%%%%%%%%%%%%%%%%%%%%%%%%%%%%%%%%%%%%%%%TITEL
%Von Oma			%%QUELLE als Kommentar


\hrulefill

Ein Rezept von Oma Lisl
\vspace*{\fill}

\begin{multicols}{2}	


Heffnkliss
\begin{itemize}
\item 500 g Mehl
\item 100 g Butter
\item 1 Pck Trockenhefe (7g)
\item 100 g Zucker
\item 100 g Butter
\item 1 Prise Salz
\item Butter zum Einfetten der Pfanne
\end{itemize}
\vfill									%%Neue//
\columnbreak								%%Spalte
Vanillesauce
\begin{itemize}
\item 1 l Milch
\item 2 Pck Vanillesauce
\item 4 EL Zucker
\item 1 Vanilleschote (optional)
\end{itemize}
\end{multicols}

\vspace{1cm}			%
\begin{center}			%
[Zutaten für ca. 4 Portionen, 1 l Vanillesauce]%%%%%%%%%%%%%%%%%%%%%%%%%%%%%%%%%%%%%%%%%%%%%%%MENGE
\end{center}


\vfill
\newpage
\textbf{{\LARGE Zubereitung}}%%%%%%%%%%%%%%%%%%%%%%%%%%%%%%%%%%%%%%%%%%%%%%%%

\hrulefill

\vspace*{\fill}
\begin{multicols}{2}
\columnseprule1pt
\textbf{Heffnkliss}\\
Butter und Milch auf Zimmertemperatur aufwärmen lassen.
Mehl, Zucker, Trockenhefe und Salz vermischen.
Die temperierte Butter unterkneten und nach und nach von der Milch zugeben, kneten bis
der Teig nicht mehr an den Händen und der Schüssel klebt.

Die Pfanne gut einfetten, den Teig in kleine Bällchen aufteilen und in der Butter rollen.
Anschliessend die Teigbällchen in einer Schicht in die Pfanne schlichten. 
Dabei etwas Platz zwischen den Bällchen lassen. Abgedeckt an einem 
warmen Ort ruhen lassen bis sich der Durchmesser der Bällchen verdoppelt hat.\\
\textbf{Bei 150$^\circ$ Ober- und Unterhitze} auf der mittleren Schiene backen.\\

Variation: Die Bällchen vor dem Einschlichten mit Marmelade füllen und/oder den Pfannenboden
mit Mandelsplittern bestreuen.\\
%Rezept von Oma

\textbf{Vanillesauce}\\
Von der Milch eine Tasse nehmen und darin das Saucenpulver und den Zucker darin glattrühren.
Die Samen aus Vanilleschote hineinrühren.
Den Rest der Milch mit der Vanilleschote darin zum Kochen bringen, dann vom Herd nehmen und das Gemisch aus Milch, 
Zucker und Saucenpulver mit dem Schneebesen einrühren.
Unter ständigem Rühren erneut kurz aufkochen.\\
Vor dem Servieren die Vanilleschote aus der Sauce fischen.



\end{multicols}
\vfill
\end{document}