\documentclass[a6paper, twoside]{report}
\usepackage[a6paper, top=7mm, inner=10mm, right=10mm, bottom=10mm, landscape]{geometry}
\usepackage[ngerman]{babel}
\usepackage[utf8]{inputenc}
\usepackage{multicol}
\pagestyle{plain}




\parindent0pt	

\pagestyle{empty}
\begin{document}

\textbf{{\LARGE Pizzabroetchen}}%%%%%%%%%%%%%%%%%%%%%%%%%%%%%%%%%%%%%%%%%%TITEL
%%%%%%%%%%%%%%%%%%%%%%%%%%%%%%%%%%%%%%%%%%%%%%%%%%%%%%%%%%%%%UNTERTITEL
%		%%QUELLE als Kommentar

\hrulefill
\vspace*{\fill}
\begin{multicols}{2}	


\begin{itemize}
\item 5		Brötchen
\item 330 g 	Sauerrahm
\item 140 g	Gouda/Emmentaler
\item 100 g	Champignons
\item 3	EL 	Olivenöl	
\item 1 	grüne Paprika
\item 2 	Tomaten
\item 10 	getrocknete Tomaten		
\item Oregano
\item Salz \& Pfeffer
\end{itemize}
\end{multicols}
\vfill									%%Neue//

\vspace{2cm}
%
\begin{center}
%
[Zutaten für 10 Portionen]%%%%%%%%%%%%%%%%%%%%%%%%%%%%%%%%%%%%%%%%%%%%%%%MENGE
\end{center}


\vfill
\newpage
\textbf{{\LARGE Zubereitung}}%%%%%%%%%%%%%%%%%%%%%%%%%%%%%%%%%%%%%%%%%%%%%%%%

\hrulefill

\vspace*{\fill}
\begin{multicols}{2}
\columnseprule1pt

Den Käse reiben, Champignons in Scheiben, Paprika und Tomaten kleinschneiden.\newline

Diese Zutaten mit dem Sauerrahm zu einer dicken Masse
vermischen.\newline


Die Brötchen halbieren, auf jedes eine getrocknete Tomate legen
und mit der Masse bestreichen.\newline


Für \textbf{ca. 10 Minuten bei 180$^\circ$ C} auf dem mittleren Backblech backen oder kalt
essen.\newline





\end{multicols}
\vfill
\end{document}