\documentclass[a6paper, twoside]{report}
\usepackage[a6paper, top=7mm, inner=10mm, right=10mm, bottom=10mm, landscape]{geometry}
\usepackage[ngerman]{babel}
\usepackage[utf8]{inputenc}
\usepackage{multicol}
\pagestyle{plain}




\parindent0pt	

\pagestyle{empty}
\begin{document}

\textbf{{\LARGE Pesto Rosso}}%%%%%%%%%%%%%%%%%%%%%%%%%%%%%%%%%%%%%%%%%%%%%%%%%%%%%%%%%%%%%%%%%%%%%%%%


\hrulefill
\vspace*{\fill}
\begin{multicols}{2}	


\begin{itemize}
\item 100 g 	getrocknete Tomaten ohne Öl
\item 400 ml 	Brühe
\item 2 Zehen 	Knoblauch
\item 3 EL 	Pinienkerne
\item $\frac{1}{2}$  	Chilischote
\item 150 ml 	Wasser
\item 25 g 	Tomatenmark
\item 100 g 	Parmesan, gerieben
\item 2 EL 	Olivenöl
\item 1 EL 	Balsamico, dunkel
\item Pfeffer 

\end{itemize}

\end{multicols}

\vspace{1cm}
\begin{center}
[Ausreichend für 4 Personen]
\end{center}

\vfill
\newpage
\textbf{{\LARGE Zubereitung}}%%%%%%%%%%%%%%%%%%%%%%%%%%%%%%%%%%%%%%%%%%%%%%%%%%%%%%%%%%%%%%%%%%%%%%%%

\hrulefill

\vspace*{\fill}
\begin{multicols}{2}
\columnseprule1pt


Tomaten in Brühe aufkochen und ca. 20 Minuten ziehen lassen.\\
 
Abgetropfte Tomaten mit Knoblauch, 2,5 EL Pinienkernen, Wasser, Tomatenmark und Essig pürieren. 
Die Chilischote entkernen, fein hacken und mit Parmesan und Olivenöl unter die Tomatenmasse ziehen. 
Mit Pfeffer abschmecken.\\

Die verbleibenden Pinienkerne zum Garnieren verwenden.



\end{multicols}
\vfill
\end{document}